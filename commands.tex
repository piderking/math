%!root = main.tex

\tcbuselibrary{breakable}

% --- Week Counter ---
\newcounter{weekcounter}
\setcounter{weekcounter}{0}

% --- Dark Theme ---
\pagecolor{black!97}
\color{white}

% --- Accent Colors ---
\definecolor{boxgray}{RGB}{40,40,40}
\definecolor{accentblue}{RGB}{90,160,220}
\definecolor{accentgreen}{RGB}{80,200,120}
\definecolor{accentorange}{RGB}{255,160,60}
\definecolor{accentborder}{RGB}{140,140,140}

% --- Week / Section ---
\newcommand{\week}[1]{%
  \refstepcounter{weekcounter}%
  \section*{Week \theweekcounter: #1}%
  \addcontentsline{toc}{section}{Week \theweekcounter: #1}%
}

% --- TMV / GPA (unnumbered but in TOC) ---
\newcommand{\tmv}[1][]{%
  \vspace{1em}%
  \noindent
  \fcolorbox{accentborder}{black!97}{\textbf{TMV: #1}}%
  \par\vspace{0.5em}%
  \addcontentsline{toc}{subsection}{TMV: #1}%
}
% --- TMV / GPA (unnumbered but in TOC) ---
\newcommand{\ps}[1][]{%
  \vspace{1em}%
  \noindent
  \fcolorbox{accentborder}{black!97}{\textbf{Problem Set: #1}}%
  \par\vspace{0.5em}%
  \addcontentsline{toc}{subsection}{Problem Set: #1}%
}
\newcommand{\gpa}[1][]{%
  \vspace{1em}%
  \noindent
  \fcolorbox{accentborder}{black!97}{\textbf{Group Assignment: #1}}%
  \par\vspace{0.5em}%
  \addcontentsline{toc}{subsection}{Group Assignment: #1}%
}

% --- Problem / Step Counters ---
\newcounter{problem}[weekcounter]
\renewcommand{\theproblem}{\theweekcounter.\arabic{problem}}

\newcounter{step}[problem]
\renewcommand{\thestep}{\theproblem.\arabic{step}}

% --- Problem Box ---
\newtcolorbox{problemBox}[1][]{
  breakable,
  colback=boxgray,
  colframe=accentblue,
  fonttitle=\bfseries,
  coltitle=white,
  title=Problem~\theproblem:~#1,
  sharp corners,
  boxrule=0.7pt,
  left=6pt,right=6pt,top=4pt,bottom=4pt,
  coltext=white
}

% --- Solution Box ---
\newtcolorbox{solutionBox}[1][]{
  breakable,
  colback=boxgray,
  colframe=accentgreen,
  fonttitle=\bfseries,
  coltitle=white,
  title=Solution,
  sharp corners,
  boxrule=0.7pt,
  left=6pt,right=6pt,top=4pt,bottom=4pt,
  coltext=white
}

% --- Step Box ---
\newtcolorbox{stepBox}[1][]{
  breakable,
  colback=boxgray,
  colframe=accentorange,
  fonttitle=\bfseries,
  coltitle=white,
  title=Step~\thestep:~#1,
  sharp corners,
  boxrule=0.7pt,
  left=6pt,right=6pt,top=4pt,bottom=4pt,
  coltext=white
}

% --- Commands ---
\newcommand{\problem}[2][]{%
  \refstepcounter{problem}%
  \addcontentsline{toc}{subsubsection}{Problem~\theproblem: #1}%
  \begin{problemBox}[#1]#2\end{problemBox}%
}

\newcommand{\solution}[1]{% 
  % \addcontentsline{toc}{paragraph}{Solution~\theproblem}%
  \begin{solutionBox}#1\end{solutionBox}%
}

\newcommand{\step}[2][]{%
  \refstepcounter{step}%
  \begin{stepBox}[#1]#2\end{stepBox}%
}

% -----------------------

\pgfkeys{
    /righttriangle/.cd,
    label A/.store in=\VertA,  % Top Vertex Label
    label B/.store in=\VertB,  % Right Vertex Label
    label C/.store in=\VertC,  % Right Angle Vertex Label
    side a/.store in=\SideA,   % Hypotenuse Label
    side b/.store in=\SideB,   % Vertical Side Label
    side c/.store in=\SideC,   % Horizontal Side Label
    angle alpha/.store in=\AngAlpha, % Bottom-right Angle Label
    angle beta/.store in=\AngBeta,   % Top-right Angle Label
    % Set initial values (defaults to empty)
    label A={}, label B={}, label C={},
    side a={}, side b={}, side c={},
    angle alpha={}, angle beta={},
}

% 3. Define the main command using xparse
\NewDocumentCommand{\righttriangle}{ m m m O{} }{
    % #1: Horizontal Leg Length (e.g., 4)
    % #2: Vertical Leg Length (e.g., 3)
    % #3: Origin Coordinate (e.g., (0,0))
    % #4: Optional Key-Value Labels (e.g., side a=5, label C=X)
    
    % 4. Parse the optional key-value arguments
    \pgfkeys{/righttriangle/.cd, #4}
    
    \begin{tikzpicture}
        % Define coordinates
        \coordinate (C) at #3; % Right Angle Vertex
        \coordinate (B) at ($(C) + (#1,0)$); % Right Vertex
        \coordinate (A) at ($(C) + (0,#2)$); % Top Vertex

        % Draw the triangle
        \draw (C) -- (B) -- (A) -- cycle;

        % Label vertices
        \node at (A) [vertex, label={above:\VertA}] {};
        \node at (B) [vertex, label={below right:\VertB}] {};
        \node at (C) [vertex, label={below left:\VertC}] {};

        % Mark the right angle
        \draw[line width=0.5pt] ($(C) + (0.2,0)$) -- ($(C) + (0.2,0.2)$) -- ($(C) + (0,0.2)$);

        % Label sides
        \path (C) -- (B) node[midway, below] {$\SideC$};
        \path (C) -- (A) node[midway, left] {$\SideB$};
        \path (A) -- (B) node[midway, above right] {$\SideA$}; % Hypotenuse label position adjusted

        % Mark and label the acute angles
        \ifx\AngAlpha\empty\else
            \pic ["$\AngAlpha$", draw, angle eccentricity=1.3, angle radius=0.8cm] {angle = A--B--C};
        \fi
        \ifx\AngBeta\empty\else
            \pic ["$\AngBeta$", draw, angle eccentricity=1.3, angle radius=0.8cm] {angle = B--A--C};
        \fi

    \end{tikzpicture}
}

